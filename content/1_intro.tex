\chapter{Einleitung}\label{ch:intro}

\section{Motivation}\label{sec:motivation}

In der heutigen digitalen Welt ist die Zuverlässigkeit und Integrität von Daten von zentraler Bedeutung. 
Täglich werden riesige Mengen an Informationen über verschiedene Kommunikationskanäle übertragen und auf unterschiedlichsten Medien gespeichert. 
Dabei ist es unvermeidlich, dass Daten durch Rauschen, physische Beschädigungen oder andere Störfaktoren verfälscht werden. 
Dies stellt eine ernsthafte Herausforderung dar, insbesondere in Bereichen wie Telekommunikation, Datenarchivierung und digitaler Medien, wo die Genauigkeit und Verfügbarkeit von Informationen entscheidend sind.
Fehlerkorrekturverfahren sind daher unverzichtbare Werkzeuge, um die Qualität und Zuverlässigkeit der übermittelten oder gespeicherten Daten sicherzustellen.

Sie reichern die zu speichernden oder zu übertragenden Daten durch den Encodierungsprozess mit Redundanz an, also zusätzliche Informationen, die zur Fehlererkennung und -korrektur dienen. Beim Decodierungsprozess wird anschließend überprüft, ob Fehler aufgetreten sind und ob diese korrigiert werden können, bevor die ursprünglichen Daten wieder verwendet werden können.

Eine besonders effektive Methode zur Fehlerkorrektur sind die Reed-Solomon-Codes.
Diese wurden 1960 von den Mathematikern Irving S. Reed und Gustave Solomon entwickelt und haben sich seither als effektive Methode zur Fehlerkorrektur etabliert. 
Reed-Solomon-Codes zeichnen sich durch ihre Fähigkeit aus, eine beträchtliche Anzahl von Fehlern zu erkennen und zu korrigieren, wodurch sie die Zuverlässigkeit von Datenübertragungen und -speicherungen erheblich verbessern.

\section{Zielsetzung der Arbeit}\label{sec:objective}

Ziel dieser Arbeit ist es, ein Verständnis der Reed-Solomon-Codes zu vermitteln. 
Obwohl diese Codes in vielen alltäglichen Technologien weit verbreitet sind, bleibt die zugrundeliegende Mathematik und konkrete Implementierung oft ein komplexes Thema. 
Diese Arbeit soll die theoretischen Grundlagen und Mechanismen hinter den Reed-Solomon-Codes systematisch darlegen und ihre praktischen Anwendungen beleuchten. 
Durch die Untersuchung der mathematischen Prinzipien und der praktischen Implementierungen soll ein tieferes Verständnis dieser wichtigen Technologie gefördert werden.

\section{Aufbau der Arbeit}\label{sec:stucture}

Nach dieser Einleitung folgt im zweiten Kapitel eine Betrachtung der Entwicklung der Reed-Solomon-Codes, die die wesentlichen Meilensteine und Durchbrüche sowie die Einordung in das Gebiet der Fehlerkorrektur-Codes darstellt. 
Kapitel drei behandelt die theoretischen Grundlagen und die Funktionsweise der Reed-Solomon-Codes, wobei die mathematischen Strukturen, Codierungs- und Decodierungsprozesse sowie spezifische Algorithmen wie der Berlekamp-Welch-Algorithmus beschrieben werden. 
Das vierte Kapitel vergleicht die Reed-Solomon-Codes mit anderen gängigen Fehlerkorrektur-Codes und beleuchtet ihre vielfältigen Anwendungsgebiete. 
Im fünften Kapitel werden die wichtigsten Erkenntnisse zusammengefasst und ein Ausblick auf zukünftige Entwicklungen gegeben.
