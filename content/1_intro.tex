\chapter{Textzusammenfassung}\label{ch:textzusammenfassung}
In dem Paper \enquote{Polynomial codes over certain finite fields} von Irving S. Reed und Gustave Solomon \cite{reedPolynomialCodesCertain1960}, welche 1960 im \enquote{Journal of the society for industrial and applied mathematics} veröffentlicht wurde, wird thematisiert, wie Polynome über endlichen Körpern zur Entwicklung von Fehlertoleranzcodes genutzt werden können. 
Dabei geht es insbesondere darum, die Grundlagen für Codierungen zu schaffen, die in der Lage sind, mehrere Fehler während der Datenübertragung oder Datenspeicherung zu korrigieren. 
Die Autoren zeigen, dass durch die Verwendung von Polynomen über bestimmten endlichen Körpern eine verbesserte Fehlertoleranz erreicht werden kann \cite[S. 300f.]{reedPolynomialCodesCertain1960}. 
In dieser Untersuchung wird nachgewiesen, inwiefern diese Codes eine effektive und zuverlässige Methode zur Fehlerkorrektur darstellen, was besonders für Anwendungen in der digitalen Kommunikation von großer Bedeutung ist \cite[S. 302]{reedPolynomialCodesCertain1960}.

Außerdem wird auch eine beispielhafte Durchführung der Codierung dargestellt, um die Theorie des Verfahrens zu verdeutlichen \cite[S. 304]{reedPolynomialCodesCertain1960}. 
Offen bleibt jedoch, wie die Performance der Codes in verschiedenen Anwendungsbereichen ausfällt, da der Algorithmus zur Dekodierung als sehr aufwendig dar gestellt wird \cite[Seite 302f.]{reedPolynomialCodesCertain1960}.
Trotz dieser offenen Fragen stellt die Arbeit eine wichtige Grundlage für das Verständnis und die Weiterentwicklung von der entwickelten Reed-Solomon-Codes dar, die in den darauf folgenden Jahren erweitert und Überarbeitet und bis heute in einigen alltäglichen Technologien Anwendung finden.