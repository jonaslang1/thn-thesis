\chapter{Problemstellung}\label{ch:intro}

Microservice-Architekturen werden zunehmend für moderne Softwareanwendungen eingesetzt und bieten eine flexible Alternative zu monolithischen Systemen.
Sie bieten zahlreiche Vorteile, darunter eine höhere Skalierbarkeit, Flexibilität und eine verbesserte Modularität, indem sie einzelne Services unabhängig voneinander bereitstellen und aktualisieren lassen.
Trotz dieser Vorteile ergeben sich jedoch einige Herausforderungen in Bezug auf die Qualitätssicherung, insbesondere im Hinblick auf die Interoperabilität zwischen den verschiedenen Services.

Eine zentrale Problemstellung liegt darin, dass Änderungen an einzelnen Microservices potenziell unerwartete Seiteneffekte auf abhängige Services haben können.
Klassische Testmethoden wie End-to-End-Tests und Integrationstests sind zwar in der Lage, solche Probleme aufzudecken, jedoch sind sie oft mit hohen Kosten, langen Ausführungszeiten und einer erhöhten Komplexität in der Wartung verbunden.
Dies macht sie insbesondere in Continuous Integration (CI)-Pipelines schwer handhabbar und ineffizient.

Contract Testing hat sich als eine potenziell leistungsfähige Alternative herausgestellt.
Es ermöglicht eine gezielte Validierung von Service-Schnittstellen, indem Verträge (Contracts) zwischen Consumer- und Provider-Services definiert und automatisch überprüft werden.
Allerdings gibt es bisher nur begrenzte wissenschaftliche Untersuchungen zur praktischen Umsetzung und Wirksamkeit dieser Methodik in CI/CD-Umgebungen.

Daher ist es essenziell, eine detaillierte Untersuchung zur Anwendung von Contract Testing in Continuous Integration Pipelines durchzuführen, um herauszufinden, ob und inwieweit diese Testmethode eine nachhaltige Verbesserung der Interoperabilität in einer Microservice-Architektur gewährleisten kann.

