\chapter{Einleitung}\label{ch:intro}

\section{Motivation}\label{sec:motivation}

In der heutigen digitalen Welt ist die Zuverlässigkeit und Integrität von Daten von zentraler Bedeutung. 
Täglich werden riesige Mengen an Informationen über verschiedene Kommunikationskanäle übertragen und auf unterschiedlichsten Medien gespeichert. 
Dabei ist es unvermeidlich, dass einige Daten durch Rauschen, physische Beschädigungen oder andere Störfaktoren verfälscht werden. 
Die Gewährleistung der Genauigkeit und Verfügbarkeit von Informationen ist insbesondere in Bereichen der Telekommunikation, Datenarchivierung und digitalen Medien von entscheidender Bedeutung. 
Die Sicherstellung dieser Faktoren stellt jedoch eine ernsthafte Herausforderung dar. 
Fehlererkennungs- und -korrekturverfahren sind daher unverzichtbare Werkzeuge, um die Qualität und Zuverlässigkeit der übermittelten oder gespeicherten Daten sicherzustellen.

Um das zu erreichen, werden bei der Kanalcodierung die zu speichernden oder zu übertragenden Daten beim Encodierungsprozess mit Redundanz, also zusätzlichen Informationen, die zur Fehlererkennung und -korrektur dienen, angereichert. 
Im Decodierungsprozess wird an Hand der Redundanz überprüft, ob Fehler aufgetreten sind und ob diese korrigiert werden können. 
Dadurch kann die Integrität der empfangenen Daten bewertet werden.

Eine besonders effektive Methode zur Fehlerkorrektur\todo{} sind die Reed-Solomon-Codes, also das ursprüngliche Verfahren und alle Weiterentwicklungen.
Die ursprüngliche Version wurde 1960 von den Mathematikern Irving S. Reed und Gustave Solomon entwickelt. 
Reed-Solomon-Codes zeichnen sich durch ihre Fähigkeit aus, so viele Fehler wie die Anzahl der hinzugefügten redundanten Informationen zu erkennen und sogar die Hälfte dieser zu korrigieren.
In dieser Arbeit liegt der Fokus auf der Fehlerkorrektur.

\section{Zielsetzung der Arbeit}\label{sec:objective}

Obwohl diese Codes in vielen alltäglichen Technologien wie CDs oder QR-Codes weit verbreitet sind, ist die zugrundeliegende Mathematik und konkrete Implementierung eher unbekannt. 
Ziel dieser Arbeit ist es, die Reed-Solomon-Codes vorzustellen und deren Funktionsweise zu erklären. 
Dazu wird nach einer kurzen historischen Betrachtung die der Kanalcodierung zu Grunde liegenden Mathematik beschrieben.
Anschließend wird die Funktionsweise des ursprünglichen Ansatzes der beiden Mathematikern und eine Weiterentwicklung, der Berlekamp-Welch-Algorithmus, erläutert.
Des Weiteren wird beleuchtet, wie diese Verfahren in der Praxis Anwendung finden. 
