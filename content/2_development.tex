\chapter{Zielsetzung}\label{ch:development}

Das Hauptziel dieser Arbeit ist die systematische Untersuchung und Evaluation von Contract Testing in Continuous Integration Pipelines, um dessen Potenzial zur Sicherstellung der Interoperabilität in einer Microservice-Architektur zu bewerten.
Dabei werden insbesondere folgende Aspekte betrachtet:

\begin{itemize}
\item Analyse verschiedener Implementierungsmöglichkeiten von Contract Testing in Continuous Integration Pipelines.
\item Untersuchung der Effektivität von Contract Testing bei der frühzeitigen Fehlererkennung und deren Auswirkungen auf die Qualitätssicherung.
\item Bewertung des Nutzens von Contract Testing hinsichtlich der langfristigen Wartbarkeit, Skalierbarkeit und Effizienz von Microservice-Architekturen.
\end{itemize}

Um diese Ziele zu erreichen, wird eine experimentelle Studie durchgeführt, bei der Contract Testing in einer realen Microservice-Anwendung integriert und evaluiert wird.
Durch den Vergleich mit traditionellen Testmethoden sollen Vor- und Nachteile von Contract Testing systematisch herausgearbeitet werden.
Abschließend werden konkrete Handlungsempfehlungen für den praktischen Einsatz in Continuous Integration Pipelines entwickelt und dokumentiert.
