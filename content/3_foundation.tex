\chapter{Vorgehensweise}\label{ch:foundation}

Die Arbeit folgt einer methodischen Vorgehensweise, die sowohl eine theoretische Analyse als auch eine praktische Evaluation umfasst.
Zunächst werden die theoretischen Grundlagen untersucht, um ein fundiertes Verständnis für Microservice-Architekturen, Continuous Integration Pipelines und Teststrategien zu schaffen.
Hierbei wird ein besonderer Fokus auf Contract Testing gelegt, indem dessen Mechanismen und Konzepte detailliert erläutert und gängige Frameworks betrachtet werden.

Anschließend erfolgt eine umfassende Analyse der Implementierungsmöglichkeiten von Contract Testing in Continuous Integration Pipelines.
Dabei werden verschiedene Ansätze untersucht und mit etablierten Testmethoden verglichen.
Die Bewertung erfolgt anhand von Kriterien wie Skalierbarkeit, Wartungsaufwand und Fehlererkennungsrate.
Zudem werden Best Practices für die Implementierung von Contract Testing identifiziert und Herausforderungen beim Einsatz dieser Methode in agilen Entwicklungsteams herausgearbeitet.

Im praktischen Teil der Arbeit wird eine experimentelle Untersuchung durchgeführt, in der Contract Testing in eine
bestehende Continuous Integration Pipeline der realen Microservice-Anwendung \enquote{msg.Sensorik} integriert
wird.
Dazu werden geeignete Frameworks ausgewählt und implementiert.
Anschließend werden gezielte Testszenarien durchgeführt, um die Auswirkungen auf den Entwicklungs- und Deployment-Prozess zu analysieren.
Durch den Vergleich der Testergebnisse mit bestehenden Teststrategien soll die Effektivität von Contract Testing bewertet werden.
Besondere Aufmerksamkeit gilt hierbei der Frage, inwiefern Contract Testing zur frühzeitigen Identifikation von Schnittstellenfehlern beiträgt und ob es eine effizientere Alternative zu traditionellen Testmethoden darstellt.

Abschließend werden die gewonnenen Erkenntnisse analysiert und konkrete Handlungsempfehlungen zur Integration von Contract Testing in Continuous Integration Pipelines abgeleitet.
Hierbei wird sowohl der potenzielle Nutzen als auch die Grenzen und Herausforderungen dieser Testmethode diskutiert.
Die Ergebnisse der Arbeit sollen dazu beitragen, Entwicklern und Architekten eine fundierte Entscheidungsgrundlage für die Implementierung von Contract Testing in ihren Softwareentwicklungsprozessen zu liefern.
