\chapter{Theoretische Grundlagen}\label{ch:foundation}

Zunächst wird die algebraische Struktur endlicher Körper, auch Galois-Felder genannt, erläutert, welche die Basis für die Encodierung und Decodierung vieler Fehlerkorrekturverfahren bildet.

\section{Endliche Körper}\label{sec:galois}

Endliche Körper sind algebraische Strukturen mit einer endlichen Anzahl von Elementen, die sowohl Addition als auch Multiplikation unterstützen. 
Ein endlicher Körper \(GF(q)\) besteht aus \(q\) Elementen, wobei \(q\) eine Potenz einer Primzahl \(p\) ist.
\[
\makebox[\linewidth]{%
	\hfill
	$q=p^{m}$\hfill
	\llap{($p\in\mathbb{P}, m\in\mathbb{N}$)}%
}
\]
Die Konstruktion eines endlichen Körpers beginnt mit der Auswahl eines irreduziblen Polynoms $p(x)$ über einem Grundkörper $GF(p)$. 
Dieses irreduzible Polynom hat die Eigenschaft, dass es nicht in Produkte niedrigergradiger Polynome zerlegt werden kann. 
Die Elemente des endlichen Körpers $GF(p^{m})$ sind die Restklassen der Polynome über $GF(p)\mod p(x)$.

Im endlichen Körper $GF(2^{m})$ beispielsweise sind die Elemente Polynome vom Grad $m-1$ mit Koeffizienten aus ${0,1}$.
Die Addition erfolgt koeffizientenweise modulo 2, während die Multiplikation durch die Multiplikation der Polynome und anschließendes Reduzieren modulo $p(x)$ bestimmt wird.

Ein entscheidendes Merkmal endlicher Körper ist das Vorhandensein eines multiplikativen Inversen für jedes nicht-null Element. 
Dies bedeutet, dass für jedes Element $a$ in $GF(q)$ ein Element $b$ existiert, sodass $a\cdot b=1$. 

Diese Eigenschaft ist wesentlich für die Implementierung der Fehlerkorrekturmechanismen der Reed-Solomon-Codes, da die Berechnungen auf endlichen Systemen durchgeführt werden.
Die Struktur dieser Felder ermöglicht die effiziente Durchführung der mathematischen Operationen, die zur Erkennung und Korrektur von Fehlern in den Daten erforderlich sind. 
