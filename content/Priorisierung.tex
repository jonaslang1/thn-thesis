\chapter{Priorisierung der Literatur}\label{ch:priorsisierung-der-literatur}
Das Paper \enquote{Polynomial codes over certain finite fields} \cite{reedPolynomialCodesCertain1960} von Irving S. Reed und Gustave Solomon ist besonders wichtig für dieses Thema, da es der ursprüngliche Artikel ist, in dem die Idee für dieses Fehlertoleranzverfahren vorgestellt wurde.
Das Paper liefert die algebraischen Grundlagen für die Konstruktion solcher Codes und untersucht die Fähigkeit zur Fehlererkennung und Fehlerkorrektur. Aus der Veröffentlichung dieses Paper resultieren alle weitere Arbeiten zu diesem Thema.

Das Buch \enquote{Reed-Solomon Codes and Their Application} \cite{wickerReedSolomonCodes1994} von Stephen B. Wicker und Vijay K. Bhargava ist relevant für das Thema, da es eine umfassende Einführung in die Praxis von Reed-Solomon-Codes bietet. 
Es werden die mathematischen Grundlagen für die praktischen Anwendungen, wie z.B. in CDs, DVDs und Satelliten- und Raumfahrtkommunikation detailliert erklärt. 
Das Buch verbindet Theorie und Praxis und zeigt, wie Reed-Solomon-Codes in auch heute noch zum Einsatz kommenden Technologien genutzt werden, um Fehlertoleranz zu gewährleisten.