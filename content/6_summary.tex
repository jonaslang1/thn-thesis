\chapter{Fazit und Ausblick}\label{ch:summary}

\section{Aktuelle Entwicklungen}

Reed-Solomon-Codes bleiben auch heute eine relevante Technologie in der Fehlerkorrektur und -detektion, obwohl sie in den letzten Jahrzehnten aus Kostengründen durch neue Methoden ergänzt und teilweise ersetzt wurden \cite{ilievAnalysisEvaluationReedSolomon2008}. 

Allerdings wird die Integration von Reed-Solomon-Codes in modernen Kom"-mu"-ni"-kations- und Speichertechnologien immer bedeutender.
Insbesondere im Bereich der drahtlosen Kommunikation und der Netzwerkcodierung werden Reed-Solomon-Codes in Kombination mit anderen Fehlerkorrekturmethoden eingesetzt, um eine höhere Zuverlässigkeit zu gewährleisten \cite[\mbox{Kapitel 1}]{conOptimalTwoDimensionalReed2024}.

\section{Zukünftige Perspektiven}

In der Zukunft werden Reed-Solomon-Codes weiterhin eine wichtige Rolle in der Datenübertragung und -speicherung spielen, insbesondere in Kombination mit anderen Technologien. 
Die Entwicklungen im Bereich der Quantenkommunikation und Quantencomputing bieten neue Möglichkeiten, in denen klassische Fehlerkorrekturverfahren wie Reed-Solomon-Codes integriert werden könnten, um Systeme zu schaffen, die auch Quanteninformationen fehlertoleranter machen \cite{grasslQuantumReedSolomonCodes1999}.

Darüber hinaus wird die steigende Nachfrage nach robusten und zuverlässigen Speichersystemen in Bereichen wie Cloud-Computing und Big Data voraussichtlich die Weiterentwicklung und Anwendung von Reed-Solomon-Codes beeinflussen. 
Die zunehmende Komplexität und Größe von Datensätzen erfordert fortschrittliche Fehlerkorrekturmechanismen, um die Integrität und Verfügbarkeit von großen Datenmengen zu gewährleisten \cite[Kapitel 5]{sathiamoorthyXORingElephantsNovel2013}.

\section{Fazit}

Reed-Solomon-Codes haben sich seit ihrer Einführung im Jahr 1960 als eine der robustesten und effektivsten Methoden zur Fehlerkorrektur und -detektion etabliert. 
Ihre Anwendung reicht von der Weltraumkommunikation über optische Datenträger bis hin zu modernen Speicher- und Kommunikationssystemen \cite{wickerReedSolomonCodes1994}. 
Trotz des Fortschritts in der Technologie und der Entwicklung neuer Fehlerkorrekturverfahren bleiben Reed-Solomon-Codes aufgrund ihrer Zuverlässigkeit ein unverzichtbares Werkzeug in vielen Anwendungsbereichen.

Die kontinuierliche Forschung und Entwicklung in diesem Bereich verspricht, die Einsatzmöglichkeiten von Reed-Solomon-Codes zu erweitern und ihre Leistungsfähigkeit zu steigern \cite{conOptimalTwoDimensionalReed2024, sippelReedSolomonCodes2019}. 
In einer zunehmend digitalisierten Welt, in der die Zuverlässigkeit und Integrität von Daten von großer Bedeutung sind, werden Reed-Solomon-Codes daher auch in Zukunft eine zentrale Rolle spielen.