\chapter{Grundlagen}\label{ch:foundation}

\section{Microservice Architektur}\label{sec:microservice_architecture}

Die Microservice-Architektur ist ein Architekturstil für die Softwareentwicklung, bei dem eine Anwendung als eine Sammlung unabhängiger, kleiner Dienste aufgebaut wird.
Jeder dieser Dienste ist eigenständig und erfüllt eine spezifische Geschäftsanforderung.
Diese Architektur hat sich als Alternative zur monolithischen Architektur etabliert, bei der alle Funktionalitäten einer Anwendung in einer einzigen, großen Codebasis zusammengefasst sind.

Ein wesentliches Merkmal der Microservice-Architektur ist die Unabhängigkeit der Dienste.
Jeder Microservice kann eigenständig entwickelt, getestet, bereitgestellt und skaliert werden.
Die lose Kopplung zwischen den Diensten wird durch wohl definierte \glspl{API} sichergestellt, die häufig auf REST
oder Messaging-Protokollen basieren~\cite{richardsonMicroservicesPatterns2018}.
Zudem wird die funktionale Trennung betont, da jeder Service einen klar definierten Anwendungsteil abdeckt und von einem eigenen Entwicklungsteam verwaltet werden kann.
Diese Struktur ermöglicht es, unterschiedliche Technologien oder Programmiersprachen für verschiedene Services einzusetzen, wodurch eine technologische Vielfalt innerhalb eines Systems entsteht.
So entstehen mehrere kleine Anwendungen, die einfacher zu verstehen und zu warten sind als ein großer
Technologie-Stack wie in einer monolithischen Architektur.

Die Microservice-Architektur bietet zahlreiche Vorteile.
Sie ermöglicht eine hohe Flexibilität, da Änderungen an einem Service unabhängig von anderen vorgenommen werden können.
Die Fehlertoleranz wird verbessert, da ein Fehler in einem Microservice nicht zwangsläufig die gesamte Anwendung beeinträchtigt.
Durch die gezielte Bereitstellung von Ressourcen für bestimmte Dienste kann die Skalierbarkeit optimiert werden.
Zudem erleichtert die überschaubare Größe einzelner Microservices deren Wartung, da der Code besser verständlich und weniger komplex ist.

Trotz ihrer Vorteile bringt die Microservice-Architektur auch Herausforderungen mit sich.
Die Kommunikation zwischen den einzelnen Diensten erfordert robuste Schnittstellen und ein zuverlässiges \gls{API}-Management, um eine reibungslose Interaktion zu gewährleisten.
Die Verwaltung von Daten kann komplex sein, da die Konsistenz zwischen verschiedenen Microservices sichergestellt
werden muss.
Darüber hinaus erfordert die Vielzahl kleiner Dienste leistungsfähige Deployment- und Monitoring-Tools, um deren Verwaltung effizient zu gestalten.

Die Microservice-Architektur bildet die Grundlage für moderne verteilte Systeme und spielt eine zentrale Rolle für
Cloud-Umgebungen.
Da bei dieser Architektur die Kommunikation zwischen den einzelnen Diensten im Vordergrund steht, ist eine zuverlässige Validierung der Schnittstellen essenziell.


\section{Arten des Softwaretests}\label{sec:testing_types}

Um die Qualität und Funktionsfähigkeit von Software sicherzustellen, kommen verschiedene Arten von Softwaretests zum Einsatz.
Diese lassen sich nach ihrem Fokus und ihrer Testtiefe kategorisieren.

Unit-Tests (Modultests) überprüfen einzelne Funktionseinheiten, wie Funktionen oder Methoden, isoliert von anderen Komponenten.
Sie dienen dazu, die korrekte Implementierung kleinster funktionaler Einheiten sicherzustellen.
Integrationstests hingegen testen das Zusammenspiel mehrerer Module oder Systeme.
Sie prüfen, ob die verschiedenen Komponenten korrekt miteinander interagieren.
Systemtests betrachten die Anwendung als Ganzes.
Dabei wird überprüft, ob das gesamte System im Einklang mit den Anforderungen funktioniert.
Akzeptanztests bilden die höchste Testebene und stellen sicher, dass die Anwendung aus Sicht der Endbenutzer den definierten Anforderungen entspricht.
Diese Tests werden häufig in enger Abstimmung mit den Stakeholdern durchgeführt~\cite{singhSoftwareTesting2011}.

Neben diesen klassischen Testarten gibt es auch spezialisierte Testmethoden wie Performance-Tests, Sicherheitstests und Usability-Tests.

Im Kontext von Microservices gewinnen automatisierte Tests und insbesondere Contract Tests an Bedeutung, da sie die Kommunikation zwischen Diensten absichern und frühzeitig Fehler in der Interaktion erkennen lassen.

\section{Testautomatisierung}\label{sec:test_automation}
